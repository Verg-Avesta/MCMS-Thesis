%%
%% This is file `mcmthesis-demo.tex',
%% generated with the docstrip utility.
%%
%% The original source files were:
%%
%% mcmthesis.dtx  (with options: `demo')
%%
%% -----------------------------------
%%
%% This is a generated file.
%%
%% Copyright (C)
%%     2010 -- 2015 by Zhaoli
%%     2014 -- 2016 by Liam 
%%     2017 -- 2019 by Xuehan
%%
%% This work may be distributed and/or modified under the
%% conditions of the LaTeX Project Public License, either version 1.3
%% of this license or (at your option) any later version.
%%
%% This work has the LPPL maintenance status `maintained'.
%%
%% The Current Maintainer of this work is Xuehan.
%%
\documentclass{mcmthesis}
\mcmsetup{CTeX = false,   % 使用 CTeX 套装时,设置为 true
        tcn = 55280, problem = A,
        sheet = true, titleinsheet = true, keywordsinsheet = true,
        titlepage = true}
\usepackage{palatino}
\usepackage{mwe}
\usepackage{graphicx}
\usepackage{subcaption}
\usepackage{float}
\usepackage{multirow}
\usepackage{indentfirst}
\usepackage{gensymb}
\usepackage[ruled,lined,commentsnumbered]{algorithm2e}
\usepackage{geometry}
\geometry{left=2cm,right=2cm,top=2cm,bottom=2cm} %%页边距

\begin{document}
\linespread{0.6} %%行间距
\setlength{\parskip}{0.5\baselineskip} %%段间距
\title{ti}

\date{\today}
	\begin{abstract}

	
		\begin{keywords}
		
		\end{keywords}
	\end{abstract}

\maketitle

\tableofcontents

\newpage

\section{Introduction}
\subsection{Problem Background}
	Since December 2019, Wuhan City, Hubei Province has continued to carry out surveillance of influenza and related diseases, and found multiple cases of viral pneumonia, all of which were diagnosed with viral pneumonia / pulmonary infection.  After analysis, the pneumonia was caused by a new coronavirus.  On January 12, 2020, the World Health Organization officially named the new coronavirus that causes the pneumonia epidemic in Wuhan as "2019-nCoV.

Due to the government's inaction in the early stage of the epidemic, people did not gradually understand the severity of the epidemic until around January 20.  At this time, because Wuhan is an important transportation hub in China, the epidemic has spread across the country.  As of 24:00 on February 7, there were 31,774 confirmed cases (including 6,101 severe cases), a total of 2,050 discharged patients, 722 deaths, a cumulative report of 34,546 confirmed cases, and 27,657 suspected cases.  A total of 345,498 close contacts were tracked, and 189,660 close contacts were still in medical observation.  The severity of the new coronavirus has far exceeded SARS.  On the evening of January 30, 2020 local time, the World Health Organization (WHO) announced that the new coronavirus epidemic was listed as a public health emergency of international concern (PHEIC).

Because of the rapid development of the modern Internet and various communication technologies, all kinds of unidentifiable news are spreading rapidly on the Internet while people are not able to distinguish them.  In order to reduce people's losses due to information confusion, an effective epidemic prediction model urgently needs to be established.
\subsection{Our Work}
	To build an effective predictive model, we need to accurately describe the different transmission patterns of the new coronavirus in the case of free spread and government intervention.

In this paper, we build a suitable model.  This model can more accurately predict the spread of the new coronavirus in China and globally in the future.  In addition, the model can also predict the results and possible losses caused by government measures, and play a certain guiding role in the formulation of government policies.

In Section 2, we state several basic assumptions.  Section 3 contains the nomenclature used inmodel statement.  Section 4 provides sufficient details about our model.  Section 5 carrys out experiment and analysis about our proposed model.  At last, we further study our model in Section 6 and make some conclusions in Section 7.
\section{Assumptions}
\section{Nomenclature}
\section{Statement of our model}
\section{Implementation}
\section{Model Analysis}
\section{Conclusion}
\newpage

\begin{appendices}


\end{appendices}
\end{document}


